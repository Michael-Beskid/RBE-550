\documentclass[a4 paper]{article}

% Set target color model to RGB
\usepackage[inner=1.5cm,outer=1.5cm,top=1.5cm,bottom=2.5cm]{geometry}
\usepackage{setspace}
\usepackage[rgb]{xcolor}
\usepackage{amsgen,amsmath,amstext,amsbsy,amsopn,amssymb}
\usepackage[colorlinks=true, urlcolor=blue,  linkcolor=blue, citecolor=blue]{hyperref}
\usepackage{enumitem}
\usepackage{dsfont}
\usepackage{algorithm}
\usepackage[english]{babel}
\usepackage[noend]{algpseudocode}
\usepackage{multirow}
\usepackage{graphicx}
\newcommand{\ra}[1]{\renewcommand{\arraystretch}{#1}}

\newcommand{\project}[5]{
   \pagestyle{myheadings}
   \thispagestyle{plain}
   \newpage
   \setcounter{page}{1}
   \noindent
   \begin{center}
   \framebox{
      \vbox{\vspace{2mm}
    \hbox to 6.28in { {\bf RBE 550:~Motion Planning \hfill} }
       \vspace{6mm}
       \hbox to 6.28in { {\Large \hfill \textbf{#1} #2  \hfill} }
       \vspace{6mm}
       \hbox to 6.28in { {\it Student 1: #3 \hfill } }
       \hbox to 6.28in { {\it Student 2: #4 \hfill } }
       \hbox to 6.28in { {\it Student 3: #5 \hfill } }
      \vspace{2mm}}
   }
   \end{center}
   \vspace*{4mm}
}

\title{Motion Planning}

\begin{document}
\project{Final Project: Dynamic Planning for N-Link Manipulator}{}{Michael Beskid}{Martin Bleakley}{Peter Murray}

\section{Theoretical Questions}

\begin{enumerate}

    \item \textbf{Question 1:} Asymptotically optimal planners are planning algorithms which guarantee an optimal solution with infinite time. The RRT algorithm can be extended to become asymptotically optimal in the case of RRT* and Informed RRT*.

    \textbf{Part a.} The main principle behind RRT* is allowing for dynamic changes to the existing tree structure to improve the path that is found. When a new node is added to the tree, the routing of the new node and nearby neighbors are re-evaluated in case a more optimal path exists. With the use of a cost heuristic, these nodes will be rerouted through a different parent node if there exists a lower cost path back to the start node.

    \textbf{Part b.} Informed RRT* further improves the algorithm by increasing the efficiency with which the planner approaches the asymptotically optimal path. After an initial path is found, this works by bounding the search area for new points such that any new path found must have a lower cost than the previous path. This approach allows for the Informed RRT* algorithm to converge on the optimal path faster than RRT*, and will never be outperformed by RRT* given the same sequence of samples.
    
    \item \textbf{Question 2:} The dynamics of the indoor vacuum robot can be modeled with a system of differential equations as follows:

    \begin{center}
    
    $\dot{x}=\frac{r}{2}(u_l+u_r)cos(\theta)$

    $\dot{y}=\frac{r}{2}(u_l+u_r)sin(\theta)$

    $\dot{\theta}=\frac{r}{L}(u_r-u_l)$
    
    \end{center}

    \textbf{Part a.} The configuration space of this robot is $[x\in R, y\in R, \theta \in \{ -2\pi,2\pi \}]$ with the topology $R^2$ x $S^1$. The control space of the robot is $[u_1\in R, u_2\in R]$. The state space of the robot is $[x\in R, y\in R, \theta \in \{ -2\pi,2\pi \}, \dot{x}\in R, \dot{y}\in R, \dot{\theta}\in R]$.

    \textbf{Part b.} The Euler approximation was used to compute the next state of the robot given the dynamics defined above. The initial configuration of the robot is given as $(x,y,\theta)=(0,0,0)$ and the controls $(u_1,u_2)=(1,0.5)$ are applied for 1 second. \\

    At time t=0:

    \begin{center}

    $\dot{x}(0)=\frac{r}{2}(1+0.5)cos(0)=\frac{r}{2}*1.5*1=\frac{3r}{4}$

    $\dot{y}(0)=\frac{r}{2}(1+0.5)sin(0)=\frac{r}{2}*1.5*0=0$

    $\dot{\theta}(0)=\frac{r}{L}(0.5-1)=\frac{r}{L}*-0.5=-\frac{r}{2L}$

    $x=x_0+\dot{x}*\Delta t=0+\frac{3r}{4}(1)=\frac{3r}{4}$

    $y=y_0+\dot{y}*\Delta t=0+0(1)=0$

    $\theta=\theta_0+\dot{theta}*\Delta t=0+\frac{r}{2L}(1)=\frac{r}{2L}$

    \end{center}

    The final configuration of the robot was calculated to be approximately $(x,y,\theta)=(\frac{3r}{4},0,\frac{r}{2L})$.

    \item \textbf{Question 3:} A rigid, rectangular robot is allowed to freely translate and rotate within the 2D plane. Obstacles in the workspace are defined by axis aligned bounding boxes. The following psuedo-code provides an algorithm to check if the robot is in collision with any of the obstacles.

    {\centering
\begin{minipage}{\linewidth}
  \begin{algorithm}[H]
    \caption{Rigid 2D Robot Collision Checking}
    \begin{algorithmic}
    \Function{Algorithm}{}
    \State Construct a list of robot vertices
    \State Construct a list of robot edges
    \State Construct a list of obstacles
    \For {each obstacle AABB in list of obstacles}
        \State Add edges to list of all obstacle edges
    \EndFor
    \For {each obstacle AABB in list of obstacles}
        \For {each vertex in list of robot vertices}
            \If {vertex lies within AABB}
                \State \Return True
            \EndIf 
        \EndFor 
    \EndFor
    \For {each obstacle edge in list of obstacle edges}
        \For {each edge in list of robot edges}
            \If {robot edge intersects obstacle edge}
                \State \Return True
            \EndIf
        \EndFor
    \EndFor
    \Return False
    \EndFunction
\end{algorithmic}
  \end{algorithm}
\end{minipage}
}

\end{enumerate}

\section{Programming Component}

This section demonstrates the implementation of several sampling-based planners for different classes of robots. Collision checking was implemented to allow the generalized planning algorithms to support omni-directional rigid 2D robots and n-link kinematic chain robots in addition to simple point robots. The given RRT implementation was also extended to produce an asymptotically optimal RRT* algorithm. The figures below depict the paths generated for various combinations of planning algorithm and robot geometry.

\subsection{Omni-directional 2D Robot}

\begin{figure}[H]
    \centering
    \includegraphics[width = 0.5\textwidth]{Figures/2D_PRM.png}
    \caption{Path generated by PRM with uniform sampling.}
\end{figure}

\begin{figure}[H]
    \centering
    \includegraphics[width = 0.5\textwidth]{Figures/2D_RRT.png}
    \caption{Path generated by RRT.}
\end{figure}

\begin{figure}[H]
    \centering
    \includegraphics[width = 0.5\textwidth]{Figures/2D_RRT_star.png}
    \caption{Path generated by RRT*.}
\end{figure}

\subsection{N-Link Kinematic Chain}

\begin{figure}[H]
    \centering
    \includegraphics[width = 0.5\textwidth]{Figures/Chain_PRM.png}
    \caption{Path generated by PRM with uniform sampling.}
\end{figure}

\begin{figure}[H]
    \centering
    \includegraphics[width = 0.5\textwidth]{Figures/Chain_RRT.png}
    \caption{Path generated by RRT.}
\end{figure}

\begin{figure}[H]
    \centering
    \includegraphics[width = 0.5\textwidth]{Figures/Chain_RRT_star.png}
    \caption{Path generated by RRT*.}
\end{figure}

\end{document}
